% In this file you should put the actual content of the blueprint.
% It will be used both by the web and the print version.
% It should *not* include the \begin{document}
%
% If you want to split the blueprint content into several files then
% the current file can be a simple sequence of \input. Otherwise It
% can start with a \section or \chapter for instance.

\chapter{}

\begin{proposition}\label{alg-hom}
    $A$を有限$K$代数とし、$h_1,\ldots,h_r:A\to K$を相異なる$K$準同型とする。
    このとき、$h:A\to K^r$を
    $$
    h(a)=(h_1(a),\ldots,h_r(a))
    $$
    により定めると、これは全射である。
    特に、$\# Hom_K(A,K)\leq\dim_KA$が成り立つ。
\end{proposition}

\begin{theorem}\label{Galois-descent1}
    有限$K$代数$A$に対し
    $$
    (L\otimes_K A)^G=A
    $$
    が成り立つ。
\end{theorem}

\begin{theorem}\label{Galois-descent2}
    \uses{finite, basis}
    有限$L$代数$B$が$G$の半線形作用を持つとする。
    このとき、$\phi:L\otimes_KB^G\to B$を$a\otimes x\mapsto ax$で定めると、
    これは$G$の半線形作用を保つ$L$同型である。
\end{theorem}

\begin{lemma}\label{finite}
    $G$は有限群である。
\end{lemma}

\begin{lemma}\label{basis}
    $L$の$K$線形空間としての基底$a_1,\ldots,a_d$をとる。
    このとき、以下を満たす$b_1,\ldots,b_d\in L$が存在する。
    $\sigma\in G$に対し、
    $$
    \sum_{i=1}^db_i\sigma(a_i)=
    \begin{cases}
        1 & (\sigma=1)\\
        0 & (\sigma\neq 1)
    \end{cases}
    $$
\end{lemma}

\begin{proposition}\label{ext-self}
    \uses{Galois-descent2}
    $G$の半線形作用を保つ$L$同型$L\otimes_KL\cong L_G$がある。
\end{proposition}

\begin{proposition}\label{subalgebra-decomp}
    \uses{alg-hom}
    $L_G$の任意の部分$L$代数$A$に対し、
    $A=L_{X_1,\ldots,X_r}$をみたす$G$の分割$$G=X_1\sqcup \cdots\sqcup X_r$$が存在する。
    さらに、このような$X_1,\ldots,X_r$は並べ替えを除いて一意である。
    すなわち、$L_G$の部分$L$代数は$G$の分割と一対一に対応する。
\end{proposition}

\begin{proposition}\label{group-decomp}
    $G$の分割$G=X_1\sqcup\cdots\sqcup X_r$に対し、以下は同値である。
    \begin{enumerate}
        \item $G=X_1\sqcup\cdots\sqcup X_r$は$G$の作用で保たれる。
        \item $G$の部分群$H$が存在して、$G=X_1\sqcup\cdots\sqcup X_r$は$G$の$H$に関する左剰余類の分割$G=g_1H\sqcup\cdots g_rH$と一致する。(このような部分群$H$は一意である。)
    \end{enumerate}
    すなわち、$G$の作用で保たれる$G$の分割は$G$の部分群と一対一に対応する。
\end{proposition}

\begin{theorem}\label{main}
    \uses{Galois-descent1, Galois-descent2, ext-self, subalgebra-decomp, group-decomp}
    $L$の部分$K$代数と$G$の部分群は一対一に対応する。
\end{theorem}