% In this file you should put the actual content of the blueprint.
% It will be used both by the web and the print version.
% It should *not* include the \begin{document}
%
% If you want to split the blueprint content into several files then
% the current file can be a simple sequence of \input. Otherwise It
% can start with a \section or \chapter for instance.

\chapter{}

\begin{proposition}\label{alg-hom}
    $A$を有限$K$代数とし、$h_1,\ldots,h_r:A\to K$を相異なる$K$準同型とする。
    このとき、$h:A\to K^r$を
    $$
    h(a)=(h_1(a),\ldots,h_r(a))
    $$
    により定めると、これは全射である。
    特に、$\# Hom_K(A,K)\leq\dim_KA$が成り立つ。
\end{proposition}

\begin{theorem}\label{Galois-descent1}
    有限$K$代数$A$に対し
    $$
    (L\otimes_K A)^G=A
    $$
    が成り立つ。
\end{theorem}

\begin{theorem}\label{Galois-descent2}
    \uses{finite, basis}
    有限$L$代数$B$が$G$の半線形作用を持つとする。
    このとき、$\phi:L\otimes_KB^G\to B$を$a\otimes x\mapsto ax$で定めると、
    これは$G$の半線形作用を保つ$L$同型である。
\end{theorem}

\begin{lemma}\label{finite}
    $G$は有限群である。
\end{lemma}

\begin{lemma}\label{basis}
    $L$の$K$線形空間としての基底$a_1,\ldots,a_d$をとる。
    このとき、以下を満たす$b_1,\ldots,b_d\in L$が存在する。
    $\sigma\in G$に対し、
    $$
    \sum_{i=1}^db_i\sigma(a_i)=
    \begin{cases}
        1 & (\sigma=1)\\
        0 & (\sigma\neq 1)
    \end{cases}
    $$
\end{lemma}

\begin{proposition}\label{ext-self}
    \uses{Galois-descent2}
    $G$の半線形作用を保つ$L$同型$L\otimes_KL\cong L_G$がある。
\end{proposition}

\begin{proposition}\label{subalgebra-decomp}
    \uses{alg-hom}
    $L_G$の任意の部分$L$代数$A$に対し、
    $A=L_{X_1,\ldots,X_r}$をみたす$G$の分割$$G=X_1\sqcup \cdots\sqcup X_r$$が存在する。
    さらに、このような$X_1,\ldots,X_r$は並べ替えを除いて一意である。
    すなわち、$L_G$の部分$L$代数は$G$の分割と一対一に対応する。
\end{proposition}

\begin{definition}\label{group-def}
    \lean{Group}
    \leanok
    集合$G$と$m:G\times G\to G, i:G\to G, e\in G$の組$(G,m,i,e)$が群であるとは
    \begin{enumerate}
        \item $m$が結合的なこと。$m(m(a,b),c)=m(a,m(b,c))$
        \item $i$が逆元であること。$m(a,i(a))=m(i(a),a)=e$
        \item $e$が単位元であること。$m(e,a)=m(a,e)=a$
    \end{enumerate}
\end{definition}

\begin{definition}\label{subgroup-def}
    \uses{group-def}
    \lean{Subgroup}
    \leanok
    群$G$の部分群$H$とは、$H\subset G$であって、
    \begin{enumerate}
        \item $H$が$m$に関して閉じている。$a,b\in H\Rightarrow m(a,b)\in H$
        \item $H$が$i$に関して閉じている。$a\in H\Rightarrow i(a)\in H$
        \item $H$が$e$を含む。$e\in H$
    \end{enumerate}
    を満たすものである。
\end{definition}

\begin{definition}\label{action-def}
    \uses{group-def}
    群$G$の集合$X$への作用とは、$a:G\times X\to X$であって、
    \begin{enumerate}
        \item $a(g,a(g',x))=a(m(a,a'),x)$
        \item $a(e,x)=x$
        \item $a(i(g),a(g,x))=x$
    \end{enumerate}
    を満たすもの。
\end{definition}

\begin{definition}
    \label{decomposition-def}
<<<<<<< HEAD
    \lean{Famiily.Decomposition}
=======
    \lean{Family.Decomposition}
>>>>>>> main
    \leanok    
    集合$X$の部分集合族$\{X_i\}_{i\in I}$が$X$の分割であるとは、
    \begin{enumerate}
        \item 任意の$i\in I$について$X_i\neq\emptyset$
        \item $X=\bigcup_{i\in I}X_i$
        \item $i\neq j\Rightarrow X_i\cap X_j=\emptyset$
    \end{enumerate}
    が成り立つことである。
\end{definition}

\begin{proposition}
    \label{left-multiplication}
    群$G$は$G$に左からの積で作用する。
    すなわち、$X=G$として$a:G\times G\to G$を$a(g,h)=gh$で定めると、これは$G$の$X$への作用である。
\end{proposition}

% 群作用による商として抽象化した方がよさそう
% 群の作用が部分群の作用を誘導することも使う
\begin{definition}
    \label{quotient-def}
    群$G$の部分群$H$に対し、左剰余類の集合を$G/H=\{gH\mid g\in G\}$と定義する。
\end{definition}

% 同値関係による分割として抽象化した方がよさそう
\begin{proposition}
    \label{subgroup-decomp}
    \uses{subgroup-def,decomposition-def,quotient-def}
    $G$の部分群$H$に対し、$G$の分割が次のように定まる。
    $f:G/H\to P(G)$を$gH\mapsto gH$で定めるとこれは分割になる。
\end{proposition}

% Xへの作用がX -> Yへの作用を誘導することも使う
\begin{proposition}
    \label{powerset-action}
    \uses{action-def}
    群$G$が$X$に作用するとき、$G$の$P(X)$への作用が次のように定まる。
    すなわち$A\in P(X)$に対し、$g\cdot A=gA=\{g\cdot x\mid x\in A\}$と定める。
    これが作用となる。
\end{proposition}

\begin{proposition}
    \label{fixed-point}
    \uses{action-def}
    群$G$が$X$に作用するとき、$x\in X$が固定点であるとは、
    任意の$g\in G$に対して$a(g,x)=g\cdot x=x$が成り立つことである。
\end{proposition}

% GがXに作用するとき、X -> Yに作用することを使う
\begin{proposition}
    \label{decomp-action}
    \uses{action-def,decomposition-def,powerset-action}
    群$G$が$X$に作用するとき、添え字集合$I$を持つ$X$の分割全体の集合にも$G$が作用する。
    すなわち$f:I\to P(X)$に対し、$g\cdot f$を$i\mapsto g\cdot f(i)$で定めるとこれは作用になる。
\end{proposition}

% 対称群の群作用による同値関係で書いた方がよさそう
\begin{proposition}
    \label{decomp-equiv, decomposition-def}
    集合$X$の分割$f:I\to X$と$g:I\to X$が同値であるとは、
    ある$\sigma\in S_I$が存在して、任意の$i\in I$に対し$f(i)=g(\sigma(i))$が成り立つことである。
\end{proposition}

\begin{proposition}\label{group-decomp}
    \uses{subgroup-def,decomposition-def,action-def,subgroup-decomp,fixed-point,decomp-action,decomp-equiv,left-multiplication,powerset-action}
    群$G$の分割$G=X_1\sqcup\cdots\sqcup X_r$に対し、以下は同値である。
    \begin{enumerate}
        \item $G=X_1\sqcup\cdots\sqcup X_r$は$G$の作用で保たれる。
        \item $G$の部分群$H$が存在して、$G=X_1\sqcup\cdots\sqcup X_r$は$G$の$H$に関する左剰余類の分割$G=g_1H\sqcup\cdots g_rH$と一致する。(このような部分群$H$は一意である。)
    \end{enumerate}
    すなわち、$G$の作用で保たれる$G$の分割は$G$の部分群と一対一に対応する。
\end{proposition}

\begin{theorem}\label{main}
    \uses{Galois-descent1, Galois-descent2, ext-self, subalgebra-decomp, group-decomp}
    $L$の部分$K$代数と$G$の部分群は一対一に対応する。
\end{theorem}